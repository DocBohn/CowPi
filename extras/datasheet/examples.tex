%! suppress = NonMatchingIf
The CowPi v\softwareversion\ library has a few example programs to demonstrate the use of the library's functions with the Cow Pi \cowpiversion.

\ifdefstring{\microcontroller}{ATmega328P}{\subsection{\texttt{stdio} Demonstration}
    The avr-libc library used by gcc when compiling programs for the ATmega328P does not normally have a couple of key \texttt{stdio} functions defined, making most of the rest of the \texttt{stdio} functions unusable.
    For most microcontroller applications, this is a reasonable decision: there usually isn't a terminal connected to an embedded system, and by omitting the \texttt{stdio} functions, the firmware uploaded to the microcontroller has a smaller footprint.
    (Consider, for example, that the ATmega328P has 32KB of flash memory for the firmware and 2KB of SRAM for the program stack and heap -- even an increase of 128 bytes is a significant fraction of the available memory.)
    The Arduino toolchain provides \function{Serial.print()} and \function{Serial.println()} to overcome the absence of \function{printf()}, and \function{Serial.readXXX()} and \function{Serial.parseXXX()} to overcome the absence of \function{scanf()}.

    If a developer is willing to accept a slightly larger binary, the \function{printf()} and \function{scanf()} offer significant benefits to application developers, including being functions that are familiar, and providing far more concise ways to read and write non-trivial inputs and outputs.
    For this reason, the CowPi library implements the missing \texttt{stdio} functions to allow application developers to choose to include \texttt{stdio} functions or not, based on their needs.

    \textbf{NOTES}
    \begin{itemize}
        \item The CowPi library's implementation of these functions includes no software buffering.
            While this keeps the memory footprint down, it does mean that \function{printf()} and \function{scanf()} are slow.
        \item As is common in most microcontroller toolchains, \function{printf()} and \function{scanf()} do not process floating point values by default.
            This keeps the memory footprint down, and it is very rare that a microcontroller application would use floating point values.
            The avr-libc library achieves this by aliasing \function{printf()} to the integer-only \function{iprintf()}.
            If you need to print or scan floating point values, you can undo the aliasing with the pre-processor directive \texttt{\#define printf}.
    \end{itemize}

    We shall now examine \textit{stdio\_demonstration.ino}:

    \lstinputlisting{../../examples/stdio_demonstration/stdio_demonstration.ino}

    The call to \hyperlink{function:cowpi_stdio_setup}{\function{cowpi_stdio_setup()}} initializes the USART connection to the serial terminal and then links the CowPi-implemented functions to avr-libc's \texttt{stdio} stubs.
    The call to \hyperlink{function:cowpi_setup}{\function{cowpi_setup()}} is provided no non-zero arguments because this particular example makes no use of peripherals.
    (A typical application program would, of course, provide useful arguments.)

    The rest of the \function{setup()} function demonstrates the use of \function{printf()}.
    The simplest way to use \function{printf()} is to use it like you would on a desktop system or server (lines 7--8).

    Due to the limited amount of SRAM, you may wish to store string constants in flash memory instead of SRAM.
    While avr-gcc will place string constants in flash memory by default microcontrollers that map instruction memory into the data memory address space, such as the ATmega4809, it does not do so for microcontrollers that have fully-distinct instruction and data address spaces, such as the ATmega328P.
    The Arduino toolchain includes the \function{F()} macro that places string constants in flash memory and generates pointers that can be used by \function{Serial.printXX()} functions (lines 10--11).
    The \function{F()} macro does not work with \function{printf()}, but a work-around is shown in lines 13--19:
    \begin{itemize}
        \item Create a buffer large enough to hold the largest string constant (here, \lstinline{char s[83]})
        \item Use the avr-libc \function{PSTR()} macro to place string constants in flash memory
        \item Use the avr-libc \function{strcpy_P()} function to copy string constants from flash memory into the buffer as needed
            \begin{itemize}
                \item \textit{\textbf{NOTE:} \function{strcpy_P()}, like \function{strcpy()}, returns a pointer to the destination buffer, which is in data memory}
            \end{itemize}
        \item Use the format specifier \lstinline{\%s} in the \function{printf()} format string to print the string at the address returned by \function{strcpy_P()}
    \end{itemize}

    The \function{loop()} function demonstrates the use of \function{scanf()}, which has no nuances other than those that are part of \function{scanf()}'s specification.
}{}

\subsection{Simple I/O Demonstration}

    This example program makes use of all the Cow Pi's peripherals except its display module.
    While this program's principal use is to test whether a mark~1 Cow Pi or a mark~2 Cow Pi has been assembled correctly (and so is named \textit{simple\_io\_test.ino}), it also demonstrates the use of the functions described in Sections~\ref{subsec:SimpleInputs}--\ref{subsec:ScannedInputs}.

    We shall now examine \textit{simple\_io\_test.ino}:

    \lstinputlisting{../../examples/simple_io_test/simple_io_test.ino}

    Because this program does not use the display module, the call to \hyperlink{function:cowpi_setup}{\function{cowpi_setup()}} does not specify a display module.
    We do, however, need to specify which serial communication protocol will be used with the display module (lines 13--14) because that affects which pins the toggleable switches are connected to for Arduino-based Cow Pi development boards.
    For the Cow Pi \cowpiversion, specify \ifdefstring{\cowpiversion}{mk3a}{whichever protocol the board is configured for}{the \ifboolexpe{bool{spi}}{SPI}{}\ifboolexpe{bool{i2c}}{I$^2$C}{} protocol}.

    The function calls on lines~27 and 35 are to \function{digitalRead()} from the Arduino toolchain.
    If you're using the CowPi library, using \function{digitalRead()} (and \function{digitalWrite()}) is generally frowned-upon.
    Much more efficient code is possible by using I/O registers (see Section~\ref{sec:IOregisterDescriptions});
    however, in this case we wanted the example to be portable across many microcontrollers.
    Directly reading individual keypad column pins or the NAND pins isn't a likely scenario in application code, so the CowPi library doesn't have functions to support it.
    (Reading the keypad column pins would be part of code that scans the keypad, and the NAND pins would be used to trigger external interrupts (see Section~\ref{sec:Interrupts})).

    The call to \hyperlink{function:cowpi_get_keypress}{\function{cowpi_get_keypress()}} on line 26 (truncated here but fully-visible in the copy available through the Arduino IDE's *Examples* menu) demonstrates that the function behaves as specified: it returns a \texttt{NUL} character when no key is pressed, and a printable character when a key is pressed.
    We see on lines 29--30, 38, and 45, that the left and right switches can be independently checked for being in the left or right position.
    Similarly, on lines 32--33, 38, and 45, we see that the left and right buttons can be independently checked to determine whether one is pressed.
    Finally, on lines 40, 43, 47, and 50, we see that the left and right LEDs can be independently illuminated and deluminated.

\ifboolexpe{bool{max7219matrix}}{\dots\dots Show max7219\_matrix\_hello\_world, pointing out the font functions \dots\dots the font functions need to be added to the software section \dots\dots}{}

\ifboolexpe{bool{max7219digits}}{\dots\dots Show max7219\_hello\_world, pointing out the font function, and max7219\_bcd\_decode\_mode \dots\dots the font functions need to be added to the software section \dots\dots}{}

\ifboolexpe{bool{lcd1602}}{
\subsection{Hello World for the LCD1602 Display Module}

    Let us now consider a simple program that uses the display module.

    \lstinputlisting{../../examples/lcd1602_hello_world/lcd1602_hello_world.ino}

    In general, there is no need to setup \texttt{stdio} (line~4); however in this case we do so that we can provide information to someone who is assembling a Cow Pi development board (lines~24--25).

    Unlike the simple I/O demonstration, since this example uses a display module, we do need to specify which display module we're using on line~22 so that the display module can be configured properly.
    We also specify whether to use SPI or I$^2$C.
    If using I$^2$C, we need to call \hyperlink{function:cowpi_set_display_i2c_address}{\function{cowpi_set_display_i2c_address()}} (line~19) before calling \function{cowpi_setup()}.
    For LCD1602 serial adapters, specifying \texttt{0x27} as the address is a very good choice, and \texttt{0x3F} is a good choice if \texttt{0x27} does not work for an unmodified serial adapter, unless you know that your adapter has been modified to use a different address.
    If \texttt{0x27} and \texttt{0x3F} don't work when assembling a Cow Pi development board, double-check the wiring;
    if the wiring is good, then run a program to scan the I$^2$C for peripherals and report their addresses.\footnote{
        Such a program is available on the Arduino website: \url{https://playground.arduino.cc/Main/I2cScanner/}.}

    Some of the functionality demonstrated in this program is:
    \begin{itemize}
        \item Using \hyperlink{function:cowpi_lcd1602_place_cursor}{\function{cowpi_lcd1602_place_cursor()}} (line~29) to position the cursor on the bottom row, column 15
        \item Using \hyperlink{function:cowpi_lcd1602_return_home}{\function{cowpi_lcd1602_return_home()}} (line~38) to position the cursor on the top row, column 0
        \item Using \hyperlink{function:cowpi_lcd1602_send_command}{\function{cowpi_lcd1602_send_command()}} (lines~30 \& 39) to change the direction that the cursor moves after placing a character
            \begin{itemize}
                \item Note that since we didn't specify whether the text should shift after placing a character, the display module defaults to the text remaining stationary on the display
            \end{itemize}
        \item Using \hyperlink{function:cowpi_lcd1602_send_character}{\function{cowpi_lcd1602_send_character()}} (lines~31, 32, 36 \& 44)
            \begin{itemize}
                \item On lines 31--32 we are obtaining the character representation of the address's hexadecimal address by adding the integer value of the character \textquotesingle 0\textquotesingle\ to each of the address's halfbyte\footnote{These lines have a bug in that they will not correctly display hex digits \texttt{A}--\texttt{F}.}
                \item Notice that on lines 31--37 we are printing a ``backwards message'' from right to left, so that it appears correct when read from left to right
                \item Note the idiom used in lines~35--37 and 43--45 to iterate over a string;
                    alternatively: \\ % for some reason, \begin{lstlisting} and \end{lstlisting} doesn't work inside the ifboolexpe block
                    \texttt{for (const char *c = the\_message; c != \textquotesingle\textbackslash 0\textquotesingle; c++) \{ \\
                    \hspace*{1cm}cowpi\_lcd1602\_send\_character(c); \\
                    \}} \\
                    (the principal advantage of the form that appears in the example program is that it would immediately stop on an empty string)
            \end{itemize}
    \end{itemize}

    The \function{loop()} function simply turns the display module's backlight on and off every half-second so that someone assembling a Cow Pi development board knows that they have successfully connected to the display module, even if they don't see any text.

\subsection{Adding Custom Characters to the LCD1602 Display Module} \label{subsec:CustomCharacters}

    Up to eight (8) custom characters can be programmed into the LCD1602 display module, corresponding to characters 0--7.
    In this program, we create three characters to produce a crude animation of a stick figure running:

    \lstinputlisting{../../examples/lcd1602_custom_characters/lcd1602_custom_characters.ino}

    We'll examine creating custom characters after looking at some other aspects of the program.
    \begin{itemize}
        \item This program uses \hyperlink{function:cowpi_lcd1602_place_character}{\function{cowpi_lcd1602_place_character()}} (lines~36, 43, \& 52) instead of \function{cowpi_lcd1602_send_character()};
            the difference is that \function{cowpi_lcd1602_send_character()} places a character at the cursor's location and then moves the cursor, whereas \function{cowpi_lcd1602_place_character()} places a character at a location specified in its arguments and does not change the cursor's location
        \item On lines~44 \& 46, we check whether the next position would be located off of the 16-column screen (column 16 is a valid off-screen column but we don't want to place our stick figure someplace that we can't see it)
        \item The custom characters are programmed into the display module with \hyperlink{function:cowpi_lcd1602_create_character}{\function{cowpi_lcd1602_create_character()}} (line~33)
    \end{itemize}

    The second argument to \function{cowpi_lcd1602_create_character()} is an 8-element array of bitvectors.
    Each bitvector corresponds to one of the eight rows in a $5 \times 8$ character matrix.
    Even though a bitvector is stored as a \texttt{uint8\_t}, only the five least-significant bits are used.
    Of these five bits, a 0 indicates that the corresponding pixel is ``off,'' and a indicates that the corresponding pixel is `on'.
    For example, consider the first element in the \lstinline{runner} matrix: \\
%    \lstinline{0x06, 0x06, 0x0c, 0x16, 0x04, 0x06, 0x09, 0x01} %% why isn't this working, either?
%    viewed in binary, that's \\
%    \lstinline{{0x06, 0x06, 0x0c, 0x16, 0x04, 0x06, 0x09, 0x01} } \\
%    \verb+{0x06, 0x06, 0x0c, 0x16, 0x04, 0x06, 0x09, 0x01}+ %% or this!?
    \texttt{\{0x06, 0x06, 0x0c, 0x16, 0x04, 0x06, 0x09, 0x01\}} \\
    Viewed in binary (and expressing only the five least-significant bits in each bitvector), that's \\
    \texttt{\{0b00110, 0b00110, 0b01100, 0b10110, 0b00100, 0b00110, 0b01001, 0b00001\}}

    We can then see one of the frames for our running stick figure by drawing an empty square wherever there's a 0 and a filled square wherever there's a 1.
    \begin{align*}
        \square\square\blacksquare\blacksquare\square       & & 0b00110 \\[-8pt]
        \square\square\blacksquare\blacksquare\square       & & 0b00110 \\[-8pt]
        \square\blacksquare\blacksquare\square\square       & & 0b01100 \\[-8pt]
        \blacksquare\square\blacksquare\blacksquare\square  & & 0b10110 \\[-8pt]
        \square\square\blacksquare\square\square            & & 0b00100 \\[-8pt]
        \square\square\blacksquare\blacksquare\square       & & 0b00110 \\[-8pt]
        \square\blacksquare\square\square\blacksquare       & & 0b01001 \\[-8pt]
        \square\square\square\square\blacksquare            & & 0b00001
    \end{align*}
}{}
