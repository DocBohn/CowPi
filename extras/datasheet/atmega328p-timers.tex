The \microcontroller\ used by the \mcuboard\ in the Cow Pi has three timers.
As with the other I/O registers, the registers used by these timers are mapped into the data memory address space.

\subsection{Structures for Memory-Mapped Timer Registers}

The CowPi library provides data structures for both 8-bit and 16-bit timers, allowing access to the memory-mapped timer registers in a more readable form.

\lstinputlisting[linerange=177-205, firstnumber=177, basicstyle=\small]{../../src/cowpi_atmega328p.h}

The \microcontroller's TIMER0 and TIMER2 are 8-bit timers, and TIMER1 is a 16-bit timer.

\textbf{NOTE:} Even though TIMER0 and TIMER2 are both 8-bit timers and make use of the same \lstinline{cowpi_timer8bit_t} structure definition, the interpretation of the bits in their fields is subtly different.
Be sure to use the correct tables when configuring the timers.

\textbf{NOTE:} The timers have uses which may not be immediately obvious.
The most critical of these is that TIMER0 is used for the pseudo-clock that allows the Arduino Core's \function{millis()} function to report the number of milliseconds since power-up.
While you can safely configure and handle comparison interrupts that do not reset the timer's counter, \textit{you should not change TIMER0's period, reset TIMER0's counter, nor register an ISR for \lstinline{TIMER0_OVF0_vect}} unless you are willing to accept the adverse impact on \function{millis()} and any code that depends on \function{millis()}.

The five fields in the structures are:

\begin{description}
    \item[\texttt{control}] The concatenation of the timer's two (TIMER0, TIMER2) or three (TIMER1) control registers
    \item[\texttt{counter}] The 8-bit (TIMER0, TIMER2) or 16-bit (TIMER1) value that increments by one in each timer period
    \item[\texttt{capture}] Stores the counter's value at the exact moment that an input capture event occurs (TIMER1 only)
    \item[\texttt{compareA}] The counter value at which a \texttt{TIMER\textit{n}\_COMPA\_vect} interrupt should be triggered, where \textit{n} is the timer number
    \item[\texttt{compareB}] The counter value at which a \texttt{TIMER\textit{n}\_COMPB\_vect} interrupt should be triggered, where \textit{n} is the timer number
\end{description}

\subsection{Configuring the Timer Period}

In a timer's ``Normal'' mode, it will increment its counter to the maximum value representable by its counter, and then overflow to zero, possibly generating a timer overflow interrupt in the process.
You can adjust the rate at which its counter increments using a prescaler that is applied to the system clock's period.
Because the number of increments is fixed, a timer has only 5 or 7 possible periods without introducing an external timer source.
In other timer modes, you have a much wider range of possible intervals between interrupts available to you, which you can obtain by setting both a prescaler and a comparison value.

Once you know the desired timer period between timer interrupts, you need to determine the comparison value and the timer prescaler that will realize this timer period.
You can then determine the parameters for the timer using
this equation:
\[
    16,000,000 \frac{\mathrm{cycles}}{\mathrm{second}} =
    comparison\_value \frac{\mathrm{beats}}{\mathrm{interrupt}} \times
    prescaler \frac{\mathrm{cycles}}{\mathrm{beat}} \times
    interrupt\_frequency \frac{\mathrm{interrupts}}{\mathrm{second}}
\]
or, equivalently:
\[
    comparison\_value \frac{\mathrm{beats}}{\mathrm{interrupt}} \times
    prescaler \frac{\mathrm{cycles}}{\mathrm{beat}} =
    16,000,000 \frac{\mathrm{cycles}}{\mathrm{second}} \times
    interrupt\_period \frac{\mathrm{seconds}}{\mathrm{interrupt}}
\]
where:
\begin{description}
    \item [16,000,000 Hz] is the system clock frequency (the inverse of the clock period).
    \item [comparison\_value] is a number you will place in one of the timer's \texttt{compare} registers for a comparison-based timer interrupt.
        This can be any possible value of an unsigned 16-bit integer for Timer1, or any possible value of an unsigned 8-bit integer for Timer2.
        (If you plan to use an overflow interrupt, then $comparison\_value$ is 256 or 65,536, depending on whether you're using an 8-bit or a 16-bit timer.)
    \item [prescaler] is a multiplier applied to the clock period to adjust the time between counter increments (``beats'').
        Possible values are 1, 8, 64, 256, and 1024 for Timer0 and Timer1, or 1, 8, 32, 64, 128, 256, and 1024 for Timer2.
    \item [interrupt\_frequency] is how often you want a timer interrupt (the inverse of the interrupt period).
    \item [interrupt\_period] is the time between timer interrupts (the inverse of the interrupt frequency).
\end{description}

You may have to iterate on your design until you arrive at one that works with the constraints of that equation's terms for whichever timer you choose to use.

If you cannot generate a timer period great enough for your needs, then follow the example used by the Arduino pseudo-clock:
Find a comparison value and a prescaler that yield a timer period that is a whole-number factor of your actual desired period.
Then introduce a counter variable in your ISR that increments each time the ISR is invoked, and when it reaches a particular value (the value at which $timer\_period \times counter\_value = desired\_period$) then take the appropriate action.

To understand the available waveform generation modes, see the \microcontroller\ datasheet;\cite{ATmega328P}
specifically, see Section~14.7 (TIMER0), Section~15.9 (TIMER1), or Section~17.7 (TIMER2).
A good choice for many applications is Clear Timer on Compare (\textit{CTC}) with the ``TOP'' value set by an output compare register (\texttt{OCRxx}, aka one of the \textit{compareX} fields in the data structure).
But, again, check the \microcontroller\ datasheet to be sure.
Use the tables below to select the appropriate WGM bits.

Based on the prescaler you chose for the above equation, use the tables below to select the appropriate CS bits.

\subsection{Configuring TIMER0}

\textbf{CAUTION: TIMER0 is used by the Arduino Core to track time since power-up.}
The Arduino Core configures TIMER0 for ``Normal'' mode with a prescaler of 64, generating a timer overflow interrupt every 1,024$\mu$s.
While it is perfectly safe to set up comparison interrupts using TIMER0, \textbf{DO NOT change TIMER0's Waveform Generation Mode or Clock Select bits, nor register an ISR for \texttt{TIMER\textit{n}\_OVF\_vect}, if your code relies upon the Arduino Core to track time.}

Table~\ref{tab:timer0registers} shows the mapping of TIMER0's registers to the \lstinline{cowpi_timer8bit_t} struct.
Creating a pointer to TIMER0's memory-mapped registers is as simple as
\begin{lstlisting}
    volatile cowpi_timer8bit_t *timer
            = (cowpi_timer8bit_t *)(cowpi_io_base + 0x24);
\end{lstlisting}
Having creating that pointer, you can access the registers using the \lstinline{cowpi_timer8bit_t}'s fields.

\begin{table}[h]
    \centering \small
    \begin{tabular}{|r|c||c|c|c|c|c|c|c|c||} \cline{1-2}
    \textbf{Name}   & \textbf{Offset}   & \multicolumn{8}{|c}{ }                                                                                                                                        \\ \hline\hline\hline
    \multicolumn{2}{|c||}{\texttt{compareB}} & \multicolumn{1}{|l}{\textbf{Bit 7}} & \multicolumn{6}{{c}}{\dots} & \multicolumn{1}{r||}{\textbf{Bit 0}}                                                 \\
    OCR0B           & 0x28              & \multicolumn{8}{|c||}{Comparison Value ``B''}                                                                                                                 \\ \hline\hline
    \multicolumn{2}{|c||}{\texttt{compareA}} & \multicolumn{1}{|l}{\textbf{Bit 7}} & \multicolumn{6}{{c}}{\dots} & \multicolumn{1}{r||}{\textbf{Bit 0}}                                                 \\
    OCR0A           & 0x27              & \multicolumn{8}{|c||}{Comparison Value ``A''}                                                                                                                 \\ \hline\hline
    \multicolumn{2}{|c||}{\texttt{counter}} & \multicolumn{1}{|l}{\textbf{Bit 7}} & \multicolumn{6}{{c}}{\dots} & \multicolumn{1}{r||}{\textbf{Bit 0}}                                                  \\
    TCNT0           & 0x26              & \multicolumn{8}{|c||}{Timer Counter Value}                                                                                                                    \\ \hline\hline
    \multicolumn{2}{|c||}{\texttt{control}} & \textbf{Bit 15} & \textbf{Bit 14} & \textbf{Bit 13}   & \textbf{Bit 12}   & \textbf{Bit 11}   & \textbf{Bit 10}   & \textbf{Bit 9}    & \textbf{Bit 8}    \\
    TCCR0B          &                   & \texttt{FOC0A}    & \texttt{FOC0B}    & \textemdash       & \textemdash       & \texttt{WGM02}    & \texttt{CS02}     & \texttt{CS01}     & \texttt{CS00}     \\ \cline{1-1}\cline{3-10}
    &                   & \textbf{Bit 7}    & \textbf{Bit 6}    & \textbf{Bit 5}    & \textbf{Bit 4}    & \textbf{Bit 3}    & \textbf{Bit 2}    & \textbf{Bit 1}    & \textbf{Bit 0}    \\
    TCCR0A          & 0x24              & \texttt{COM0A1}   & \texttt{COM0A0}   & \texttt{COM0BA1}  & \texttt{COM0B0}   & \textemdash       & \textemdash       & \texttt{WGM01}    & \texttt{WGM00}    \\ \hline
    \end{tabular}
    \caption{Timer0's registers. \tiny Adapted from ATmega328P Data Sheet, §14.9.\cite{ATmega328P} \label{tab:timer0registers}}
\end{table}

The two comparison values, which you can set, are continuously compared to the timer's counter value.
A comparison match can be used to generate an output compare interrupt (\lstinline{TIMER0_COMPA_vect} or \lstinline{TIMER0_COMPB_vect}).
The timer's counter value can be read from or written to by your program.
Polling the counter value is a notional use case, but configuring an interrupt would be more appropriate.
Assigning a value, such as 0, to the counter would be a mechanism to reset its counter to a known value.

Among the bits in the \lstinline{control} field (the \texttt{TCCR0A} \& \texttt{TCCR0B} registers), most can be left as 0.
If you believe that you need to set custom ``Force Output Compare'' ``Compare Output Mode'' bits, then consult the ATmega328P datasheet, Section~14.9.\cite{ATmega328P}
Under typical Cow Pi usage, you should only need to set the ``Waveform Generation Mode'' and ``Clock Select'' bits.

Using the prescaler that you determined above, you should assign the \texttt{CS00}, \texttt{CS01}, and \texttt{CS02} bits using Table~\ref{tab:timer0clockselect}.

\begin{table}[h]
    \centering \small
    \begin{tabular}{|c|c|c|l|} \hline
    \textbf{CS02}   & \textbf{CS01} & \textbf{CS00} & \textbf{Description}                      \\ \hline\hline
    0               & 0             & 0             & No clock source (Timer/Counter stopped)   \\ \hline
    0               & 0             & 1             & $\frac{clk}{1}$ (no prescaling)           \\ \hline
    0               & 1             & 0             & $\frac{clk}{8}$ (from prescaler)          \\ \hline
    0               & 1             & 1             & $\frac{clk}{64}$ (from prescaler)         \\ \hline
    1               & 0             & 0             & $\frac{clk}{256}$ (from prescaler)        \\ \hline
    1               & 0             & 1             & $\frac{clk}{1024}$ (from prescaler)       \\ \hline
    \end{tabular}
    \caption{Timer0's Clock Select Bit Description. \tiny Abridged from ATmega328P Data Sheet, Table~14\mbox{-}9.\cite{ATmega328P} See the original table for the clock select bits when using an external clock source. \label{tab:timer0clockselect}}
\end{table}

The Waveform Generation Bits are used to set the Timer/Counter mode of operation.
There are two modes most useful for typical Cow Pi usage.
The first is ``Normal'' mode, in which the counter increases monotonically until it reaches the greatest possible representable value and then overflows to 0.
The other mode is ``Clear Timer on Compare'' (\textit{CTC}) with the ``TOP'' value set by output compare register ``A,'' in which the counter increases monotonically until it reaches the value in the comparison register and then resets to 0.
The \texttt{WGM} bits for these two modes are shown in Table~\ref{tab:timer0wgm}.
For the Pulse Width Modulation modes, consult Section~14.7 and Table~14\mbox{-}8 of the ATmega328P datasheet.\cite{ATmega328P}.

\begin{table}[h]
    \centering \small
    \begin{tabular}{|c|c|c|c|c|} \hline
    \textbf{WGM02}    & \textbf{WGM01}    & \textbf{WGM00}    & \textbf{Timer/Counter Mode of Operation}  & TOP    \\ \hline\hline
    0                 & 0                 & 0                 & Normal                                    & 0xFF   \\ \hline
    0                 & 1                 & 0                 & CTC                                       & OCR2A  \\ \hline
    \end{tabular}
    \caption{Timer0's Waveform Generation Mode Bit Description. \tiny Abridged from ATmega328P Data Sheet, Table~14\mbox{-}8.\cite{ATmega328P} See the original table for the WGM bits when using a PWM mode, and for the ``OCRx [sic] Update'' and ``TOV [sic] Flag Set'' columns. \label{tab:timer0wgm}}
\end{table}

After configuring the timer, enable the relevant interrupt(s) as described in Section~\ref{subsec:timerInterrupts}, and register any necessary ISRs as described in Section~\ref{subsec:isr}.


\subsection{Configuring TIMER1}

Table~\ref{tab:timer1registers} shows the mapping of TIMER1's registers to the \lstinline{cowpi_timer16bit_t} struct.
Creating a pointer to TIMER1's memory-mapped registers is as simple as
\begin{lstlisting}
    volatile cowpi_timer16bit_t *timer
            = (cowpi_timer16bit_t *)(cowpi_io_base + 0x60);
\end{lstlisting}
Having creating that pointer, you can access the registers using the \lstinline{cowpi_timer16bit_t}'s fields.

\begin{table}[h]
    \centering \small
    \begin{tabular}{|r|c||c|c|c|c|c|c|c|c||} \cline{1-2}
        \textbf{Name}   & \textbf{Offset}   & \multicolumn{8}{|c}{ }                                                                                                                                        \\ \hline\hline\hline
        \multicolumn{2}{|c||}{\texttt{compareB}} & \multicolumn{1}{|l}{\textbf{Bit 15}} & \multicolumn{6}{{c}}{\dots} & \multicolumn{1}{r||}{\textbf{Bit 0}}                                                \\
        OCR1B           & 0x6A              & \multicolumn{8}{|c||}{Comparison Value ``B''}                                                                                                                 \\ \hline\hline
        \multicolumn{2}{|c||}{\texttt{compareA}} & \multicolumn{1}{|l}{\textbf{Bit 15}} & \multicolumn{6}{{c}}{\dots} & \multicolumn{1}{r||}{\textbf{Bit 0}}                                                \\
        OCR1A           & 0x68              & \multicolumn{8}{|c||}{Comparison Value ``A''}                                                                                                                 \\ \hline\hline
        \multicolumn{2}{|c||}{\texttt{capture}} & \multicolumn{1}{|l}{\textbf{Bit 15}} & \multicolumn{6}{{c}}{\dots} & \multicolumn{1}{r||}{\textbf{Bit 0}}                                                 \\
        ICR1            & 0x66              & \multicolumn{8}{|c||}{Input Capture's Counter Value}                                                                                                          \\ \hline\hline
        \multicolumn{2}{|c||}{\texttt{counter}} & \multicolumn{1}{|l}{\textbf{Bit 15}} & \multicolumn{6}{{c}}{\dots} & \multicolumn{1}{r||}{\textbf{Bit 0}}                                                 \\
        TCNT1           & 0x64              & \multicolumn{8}{|c||}{Timer Counter Value}                                                                                                                    \\ \hline\hline
        \multicolumn{2}{|c||}{\texttt{control}} & \textbf{Bit 23} & \textbf{Bit 22} & \textbf{Bit 21}   & \textbf{Bit 20}   & \textbf{Bit 19}   & \textbf{Bit 18}   & \textbf{Bit 17}   & \textbf{Bit 16}   \\
        TCCR1C          &                   & \texttt{FOC1A}    & \texttt{FOC1B}    & \textemdash       & \textemdash       & \textemdash       & \textemdash       & \textemdash       & \textemdash       \\ \cline{1-1}\cline{3-10}
                        &                   & \textbf{Bit 15}   & \textbf{Bit 14}   & \textbf{Bit 13}   & \textbf{Bit 12}   & \textbf{Bit 11}   & \textbf{Bit 10}   & \textbf{Bit 9}    & \textbf{Bit 8}    \\
        TCCR1B          &                   & \texttt{ICNC1}    & \texttt{ICES1}    & \textemdash       & \texttt{WGM13}    & \texttt{WGM12}    & \texttt{CS12}     & \texttt{CS11}     & \texttt{CS10}     \\ \cline{1-1}\cline{3-10}
                        &                   & \textbf{Bit 7}    & \textbf{Bit 6}    & \textbf{Bit 5}    & \textbf{Bit 4}    & \textbf{Bit 3}    & \textbf{Bit 2}    & \textbf{Bit 1}    & \textbf{Bit 0}    \\
        TCCR1A          & 0x60              & \texttt{COM1A1}   & \texttt{COM1A0}   & \texttt{COM1BA1}  & \texttt{COM1B0}   & \textemdash       & \textemdash       & \texttt{WGM11}    & \texttt{WGM10}    \\ \hline
    \end{tabular}
    \caption{Timer1's registers. \tiny Adapted from ATmega328P Data Sheet, §15.11.\cite{ATmega328P} \label{tab:timer1registers}}
\end{table}

The two comparison values, which you can set, are continuously compared to the timer's counter value.
A comparison match can be used to generate an output compare interrupt (\lstinline{TIMER1_COMPA_vect} or \lstinline{TIMER1_COMPB_vect}).
Input capture is beyond the scope of typical Cow Pi usage;
see the ATmega328P datasheet, Section~15.6\cite{ATmega328P} for discussion of the input capture unit.
The timer's counter value can be read from or written to by your program.
Polling the counter value is a notional use case, but configuring an interrupt would be more appropriate.
Assigning a value, such as 0, to the counter would be a mechanism to reset its counter to a known value.

Among the bits in the \lstinline{control} field (the \texttt{TCCR1A}, \texttt{TCCR1B}, \& \texttt{TCCR1C} registers), most can be left as 0.
If you believe that you need to set custom ``Force Output Compare,'' ``Input Capture,'' or ``Compare Output Mode'' bits, then consult the ATmega328P datasheet, Section~15.11.\cite{ATmega328P}
Under typical Cow Pi usage, you should only need to set the ``Waveform Generation Mode'' and ``Clock Select'' bits.

Using the prescaler that you determined above, you should assign the \texttt{CS10}, \texttt{CS11}, and \texttt{CS12} bits using Table~\ref{tab:timer1clockselect}.

\begin{table}[h]
    \centering \small
    \begin{tabular}{|c|c|c|l|} \hline
    \textbf{CS12}   & \textbf{CS11} & \textbf{CS10} & \textbf{Description}                          \\ \hline\hline
        0               & 0             & 0             & No clock source (Timer/Counter stopped)   \\ \hline
        0               & 0             & 1             & $\frac{clk}{1}$ (no prescaling)           \\ \hline
        0               & 1             & 0             & $\frac{clk}{8}$ (from prescaler)          \\ \hline
        0               & 1             & 1             & $\frac{clk}{64}$ (from prescaler)         \\ \hline
        1               & 0             & 0             & $\frac{clk}{256}$ (from prescaler)        \\ \hline
        1               & 0             & 1             & $\frac{clk}{1024}$ (from prescaler)       \\ \hline
    \end{tabular}
    \caption{Timer1's Clock Select Bit Description. \tiny Abridged from ATmega328P Data Sheet, Table~15\mbox{-}6.\cite{ATmega328P} See the original table for the clock select bits when using an external clock source. \label{tab:timer1clockselect}}
\end{table}

The Waveform Generation Bits are used to set the Timer/Counter mode of operation.
There are two modes most useful for typical Cow Pi usage.
The first is ``Normal'' mode, in which the counter increases monotonically until it reaches the greatest possible representable value and then overflows to 0.
The other mode is ``Clear Timer on Compare'' (\textit{CTC}) with the ``TOP'' value set by output compare register ``A,'' in which the counter increases monotonically until it reaches the value in the comparison register and then resets to 0.
The \texttt{WGM} bits for these two modes are shown in Table~\ref{tab:timer1wgm}.
For the Pulse Width Modulation modes, consult Section~15.9 and Table~15\mbox{-}5 of the ATmega328P datasheet.\cite{ATmega328P}.

\begin{table}[h]
    \centering \small
    \begin{tabular}{|c|c|c|c|c|c|} \hline
    \textbf{WGM13}  & \textbf{WGM12}    & \textbf{WGM11}    & \textbf{WGM10}    & \textbf{Timer/Counter Mode of Operation}  & TOP    \\ \hline\hline
    0               & 0                 & 0                 & 0                 & Normal                                    & 0xFFFF \\ \hline
    0               & 1                 & 0                 & 0                 & CTC                                       & OCR1A  \\ \hline
    \end{tabular}
    \caption{Timer1's Waveform Generation Mode Bit Description. \tiny Abridged from ATmega328P Data Sheet, Table~15\mbox{-}5.\cite{ATmega328P} See the original table for the WGM bits when using a PWM mode, and for the ``OCR1x Update'' and ``TOV1 Flag Set'' columns. \label{tab:timer1wgm}}
\end{table}

After configuring the timer, enable the relevant interrupt(s) as described in Section~\ref{subsec:timerInterrupts}, and register any necessary ISRs as described in Section~\ref{subsec:isr}.

\subsection{Configuring TIMER2}
Table~\ref{tab:timer2registers} shows the mapping of TIMER2's registers to the \lstinline{cowpi_timer8bit_t} struct.
Creating a pointer to TIMER2's memory-mapped registers is as simple as
\begin{lstlisting}
    volatile cowpi_timer8bit_t *timer
            = (cowpi_timer8bit_t *)(cowpi_io_base + 0x90);
\end{lstlisting}
Having creating that pointer, you can access the registers using the \lstinline{cowpi_timer8bit_t}'s fields.

\begin{table}[h]
    \centering \small
    \begin{tabular}{|r|c||c|c|c|c|c|c|c|c||} \cline{1-2}
    \textbf{Name}   & \textbf{Offset}   & \multicolumn{8}{|c}{ }                                                                                                                                        \\ \hline\hline\hline
    \multicolumn{2}{|c||}{\texttt{compareB}} & \multicolumn{1}{|l}{\textbf{Bit 7}} & \multicolumn{6}{{c}}{\dots} & \multicolumn{1}{r||}{\textbf{Bit 0}}                                                 \\
    OCR2B           & 0x94              & \multicolumn{8}{|c||}{Comparison Value ``B''}                                                                                                                 \\ \hline\hline
    \multicolumn{2}{|c||}{\texttt{compareA}} & \multicolumn{1}{|l}{\textbf{Bit 7}} & \multicolumn{6}{{c}}{\dots} & \multicolumn{1}{r||}{\textbf{Bit 0}}                                                 \\
    OCR2A           & 0x93              & \multicolumn{8}{|c||}{Comparison Value ``A''}                                                                                                                 \\ \hline\hline
    \multicolumn{2}{|c||}{\texttt{counter}} & \multicolumn{1}{|l}{\textbf{Bit 7}} & \multicolumn{6}{{c}}{\dots} & \multicolumn{1}{r||}{\textbf{Bit 0}}                                                  \\
    TCNT2           & 0x92              & \multicolumn{8}{|c||}{Timer Counter Value}                                                                                                                    \\ \hline\hline
    \multicolumn{2}{|c||}{\texttt{control}} & \textbf{Bit 15} & \textbf{Bit 14} & \textbf{Bit 13}   & \textbf{Bit 12}   & \textbf{Bit 11}   & \textbf{Bit 10}   & \textbf{Bit 9}    & \textbf{Bit 8}    \\
    TCCR2B          &                   & \texttt{FOC2A}    & \texttt{FOC2B}    & \textemdash       & \textemdash       & \texttt{WGM22}    & \texttt{CS22}     & \texttt{CS21}     & \texttt{CS20}     \\ \cline{1-1}\cline{3-10}
                    &                   & \textbf{Bit 7}    & \textbf{Bit 6}    & \textbf{Bit 5}    & \textbf{Bit 4}    & \textbf{Bit 3}    & \textbf{Bit 2}    & \textbf{Bit 1}    & \textbf{Bit 0}    \\
    TCCR2A          & 0x90              & \texttt{COM2A1}   & \texttt{COM2A0}   & \texttt{COM2BA1}  & \texttt{COM2B0}   & \textemdash       & \textemdash       & \texttt{WGM21}    & \texttt{WGM20}    \\ \hline
    \end{tabular}
    \caption{Timer2's registers. \tiny Adapted from ATmega328P Data Sheet, §17.11.\cite{ATmega328P} \label{tab:timer2registers}}
\end{table}

The two comparison values, which you can set, are continuously compared to the timer's counter value.
A comparison match can be used to generate an output compare interrupt (\lstinline{TIMER2_COMPA_vect} or \lstinline{TIMER2_COMPB_vect}).
The timer's counter value can be read from or written to by your program.
Polling the counter value is a notional use case, but configuring an interrupt would be more appropriate.
Assigning a value, such as 0, to the counter would be a mechanism to reset its counter to a known value.

Among the bits in the \lstinline{control} field (the \texttt{TCCR2A} \& \texttt{TCCR2B} registers), most can be left as 0.
If you believe that you need to set custom ``Force Output Compare'' ``Compare Output Mode'' bits, then consult the ATmega328P datasheet, Section~17.11.\cite{ATmega328P}
Under typical Cow Pi usage, you should only need to set the ``Waveform Generation Mode'' and ``Clock Select'' bits.

Using the prescaler that you determined above, you should assign the \texttt{CS20}, \texttt{CS21}, and \texttt{CS22} bits using Table~\ref{tab:timer2clockselect}.

\begin{table}[h]
    \centering \small
    \begin{tabular}{|c|c|c|l|} \hline
    \textbf{CS22}   & \textbf{CS21} & \textbf{CS20} & \textbf{Description}                      \\ \hline\hline
    0               & 0             & 0             & No clock source (Timer/Counter stopped)   \\ \hline
    0               & 0             & 1             & $\frac{clk}{1}$ (no prescaling)           \\ \hline
    0               & 1             & 0             & $\frac{clk}{8}$ (from prescaler)          \\ \hline
    0               & 1             & 1             & $\frac{clk}{32}$ (from prescaler)         \\ \hline
    1               & 0             & 0             & $\frac{clk}{64}$ (from prescaler)         \\ \hline
    1               & 0             & 1             & $\frac{clk}{128}$ (from prescaler)        \\ \hline
    1               & 1             & 0             & $\frac{clk}{256}$ (from prescaler)        \\ \hline
    1               & 1             & 1             & $\frac{clk}{1024}$ (from prescaler)       \\ \hline
    \end{tabular}
    \caption{Timer2's Clock Select Bit Description. \tiny Copied from ATmega328P Data Sheet, Table~17\mbox{-}9.\cite{ATmega328P} \label{tab:timer2clockselect}}
\end{table}

The Waveform Generation Bits are used to set the Timer/Counter mode of operation.
There are two modes most useful for typical Cow Pi usage.
The first is ``Normal'' mode, in which the counter increases monotonically until it reaches the greatest possible representable value and then overflows to 0.
The other mode is ``Clear Timer on Compare'' (\textit{CTC}) with the ``TOP'' value set by output compare register ``A,'' in which the counter increases monotonically until it reaches the value in the comparison register and then resets to 0.
The \texttt{WGM} bits for these two modes are shown in Table~\ref{tab:timer2wgm}.
For the Pulse Width Modulation modes, consult Section~17.7 and Table~17\mbox{-}8 of the ATmega328P datasheet.\cite{ATmega328P}.

\begin{table}[h]
    \centering \small
    \begin{tabular}{|c|c|c|c|c|} \hline
    \textbf{WGM22}    & \textbf{WGM21}    & \textbf{WGM20}    & \textbf{Timer/Counter Mode of Operation}  & TOP    \\ \hline\hline
    0                 & 0                 & 0                 & Normal                                    & 0xFF   \\ \hline
    0                 & 1                 & 0                 & CTC                                       & OCR2A  \\ \hline
    \end{tabular}
    \caption{Timer1's Waveform Generation Mode Bit Description. \tiny Abridged from ATmega328P Data Sheet, Table~17\mbox{-}8.\cite{ATmega328P} See the original table for the WGM bits when using a PWM mode, and for the ``OCRx [sic] Update'' and ``TOV [sic] Flag Set'' columns. \label{tab:timer2wgm}}
\end{table}

After configuring the timer, enable the relevant interrupt(s) as described in Section~\ref{subsec:timerInterrupts}, and register any necessary ISRs as described in Section~\ref{subsec:isr}.


\subsection{Timer Interrupts} \label{subsec:timerInterrupts}

Table~\ref{tab:timerInterrupt} shows the bits of the Timer Interrupt Mask registers.
To access these as memory-mapped registers, create a \lstinline{uint8_t} pointer, and assign that pointer to the lowest address of these registers:

\begin{lstlisting}
    volatile uint8_t *timer_interrupt_masks = cowpi_io_base + 0x4E;
\end{lstlisting}

This pointer can then be used as a 3-element array, indexed by the timer number.
For example, \lstinline{timer_interrupt_masks[0]} can be used to enable any of the TIMER0 interrupts.

\begin{table}[h]
    \centering \footnotesize
    \begin{tabular}{|r|c||c|c|c|c|c|c|c|c|} \hline
        \textbf{Name}   & \textbf{Offset}   & \textbf{Bit 7}    & \textbf{Bit 6}    & \textbf{Bit 5}    & \textbf{Bit 4}    & \textbf{Bit 3}    & \textbf{Bit 2}    & \textbf{Bit 1}    & \textbf{Bit 0}    \\ \hline\hline
        TIMSK2          & 0x50              & \textemdash       & \textemdash       & \textemdash       & \textemdash       & \textemdash        & OCIE2B           & OCIE2A            & TOIE2             \\ \hline
        TIMSK1          & 0x4F              & \textemdash       & \textemdash       & ICIE1             & \textemdash       & \textemdash        & OCIE1B           & OCIE1A            & TOIE1             \\ \hline
        TIMSK0          & 0x4E              & \textemdash       & \textemdash       & \textemdash       & \textemdash       & \textemdash        & OCIE0B           & OCIE0A            & TOIE0             \\ \hline
    \end{tabular}
    \caption{Timer Interrupt Mask registers. \tiny Original data from ATmega328P datasheet, §30.\cite{ATmega328P} \label{tab:timerInterrupt}}
\end{table}

For Timer Overflow interrupts, to enable \texttt{TIMER\textit{n}\_OVF\_vect}, set the \texttt{TOIE\textit{n}} bit to 1.
For Timer Comparison interrupts, to enable \texttt{TIMER\textit{n}\_COMPA\_vect}, set the \texttt{OCIE\textit{n}A} bit to 1;
to enable \texttt{TIMER\textit{n}\_COMPB\_vect}, set the \texttt{OCIE\textit{n}B} bit to 1.
For Input Capture interrupts, to enable \texttt{TIMER1\_CAPT\_vect}, set the \texttt{ICIE1} bit.

After enabling timer interrupts, be sure to register any necessary ISRs as described in Section~\ref{subsec:isr}.
