Most interrupts on the \microcontroller\ are handled either by creating an interrupt service routine (ISR) using AVR-libc's \function{ISR()} macro.
External interrupts, however, can be handled either by creating an ISR or by using the Arduino Core's \function{attachInterrupt()} function to register an interrupt handler.


\subsection{Sharing data with ISRs and Interrupt Handlers} \label{subsec:isr}

Regardless of whether you create an ISR using the \function{ISR()} macro or register an interrupt handler using the \function{attachInterrupt()} function, data cannot be passed to the interrupt-handling code through parameters, and the interrupt-handling code cannot return data through a return value.
This necessitates the use of global variables to provide data to, and obtain data from, the interrupt-handling code.

Because the compiler cannot detect any definition-use pairs for these global variables --
they are updated in one function and read in another, and no call chain exists between the two functions --
the compiler will optimize-away these variables and the code that accesses them in the interest of reducing the program's memory footprint.
The way to prevent this mis-optimization is to use the \lstinline{volatile} keyword.

\textit{Any global variables that interrupt-handling code reads from and/or writes to \textbf{must} have the \lstinline{volatile} modifier.}


\subsection{Registering Interrupt Service Routines using the \function{ISR()} Macro} \label{subsec:isrMacro}

To create an interrupt service routine, write this code that looks like a function, outside of any other function:

\begin{lstlisting}
ISR(vector) {
    ...
}
\end{lstlisting}

where $vector$ is one of the vectors listed in the AVR-libc's interrupts documentation\cite{avrInterrupt}
(look in the table rows that have ``ATmega328P'' in the ``Applicable for Device'' cell).
Replace ``\dots'' with the code that should execute whenever the timer interrupt occurs.
You want to keep your ISR short, no more than a few lines of code.
If anything more elaborate needs to happen, code in your \function{loop()} function (or a function called by \function{loop()}) can do that based on changes made from within your ISR.

Any necessary configuration to establish the conditions under which the ISR will be invoked, typically through the use of memory-mapped I/O registers, will need to occur either in the \function{setup()} function or in a helper function called by \function{setup()}.

\subsection{Registering External Interrupt Handlers using \function{attachInterrupt()}}

The \microcontroller\ has two types of interrupts that are based on changes detected at the pins.

\textit{Pin Change Interrupts} are the less flexible of the two but can be triggered by changes on any of the digital pins.
See the \microcontroller\ datasheet\cite{ATmega328P} for details on configuring pin change interrupts.
Pin change interrupts must be handled through an ISR, as described above.

\textit{External Interrupts} can also be manually configured\cite{ATmega328P} and handled through an ISR;
however, the Arduino Core has functions that abstract-away all of the configuration details.\cite{arduinoInterrupt}
While external interrupts on the \microcontroller\ are limited to only two pins (digital pins D2 \& D3), they can be triggered for many conditions, and their interrupt handlers can be easily registered.
\textit{For this reason, a NAND of the pushbuttons is input to pin D2, and a NAND of the keypad's columns is input to pin D3.}

To handle an interrupt, first write a function, such as \function{handle_buttonpress()} or \function{handle_keypress()}.
This function must not have any parameters, and its return type must be \lstinline{void}.
Then, in the \function{setup()} function (or in one of its helper functions), register the interrupt handler with this code:
\begin{lstlisting}[basicstyle=\small]
attachInterrupt(digitalPinToInterrupt(pin_number), interrupt_handler_name, mode);
\end{lstlisting}
This will configure all of the necessary registers to call the function \function{interrupt_handler_name()} whenever the input value on the pin \textit{pin\_number} satisfies the \textit{mode}.
The \textit{mode} is one of:
\begin{description}
    \item [LOW] to trigger the interrupt whenever the pin is low
    \item [RISING] to trigger the interrupt whenever the pin goes from low to
    high
    \item [FALLING] to trigger the interrupt whenever the pin goes from high to
    low
    \item [CHANGE] to trigger the interrupt whenever the pin rises or falls
\end{description}

As with ISRs registered with the \function{ISR()} macro, you want to keep your interrupt handler short.

